\documentclass[1pt]{article}
\usepackage{geometry}                % See geometry.pdf to learn the layout options. There are lots.
\geometry{letterpaper}                   % ... or a4paper or a5paper or ... 
%\geometry{landscape}                % Activate for for rotated page geometry
%\usepackage[parfill]{parskip}    % Activate to begin paragraphs with an empty line rather than an indent
\usepackage{graphicx}
\usepackage{amssymb}
\usepackage{epstopdf}
\usepackage[utf8]{inputenc}
\usepackage{graphicx}
\usepackage{float}
\usepackage{amsmath,amsthm}
\usepackage{caption}% http://ctan.org/pkg/caption
\usepackage{color}
\usepackage{hyperref,}
\usepackage{cleveref}

\DeclareGraphicsRule{.tif}{png}{.png}{`convert #1 `dirname #1`/`basename #1 .tif`.png}

\newcommand{\be}{\begin{equation}}
\newcommand{\ee}{\end{equation}}

\newcommand{\bea}{\begin{align}}
\newcommand{\eea}{\end{align}}

\newcommand{\IR}{\mathbb{R}}
\newcommand{\IN}{\mathbb{N}}


\newcommand{\eps}{\varepsilon}
\newcommand{\bu}{\mathbf{u}}
\newcommand{\bx}{{\bf x}}
\newcommand{\bX}{{\bf X}}
\newcommand{\bp}{{\bf p}}
\newcommand{\by}{{\bf y}}
\newcommand{\bA}{{\bf A}}
\newcommand{\bC}{{\bf C}}
\newcommand{\bI}{{\bf I}}
\newcommand{\bR}{{\bf R}}
\newcommand{\bP}{{\bf P}}
\newcommand{\bZ}{{\bf Z}}
\newcommand{\bT}{{\bf T}}

\newcommand{\tbA}{\tilde{\bA}}
\newcommand{\ba}{{\bf a}}
\newcommand{\bb}{{\bf b}}
\newcommand{\bn}{{\bf n}}
\newcommand{\bB}{{\bf B}}
\newcommand{\bY}{{\bf Y}}
\newcommand{\bz}{{\bf z}}
\newcommand{\bv}{{\bf v}}
\newcommand{\bw}{{\bf w}}
\newcommand{\bV}{{\bf V}}
\newcommand{\bDT}{{\bf DT}}
\newcommand{\bDV}{{\bf DV}}
\newcommand{\bzp}{{\bf z'}}
\newcommand{\bl}{{\pmb \lambda}}
\newcommand{\bmu}{\bm \mu} 
\newcommand{\bdelta}{\mathbf{\delta}}


\newcommand{\mA}{\mathcal{A}}
\newcommand{\mD}{\mathcal{D}}
\newcommand{\mH}{\mathcal{H}}
\newcommand{\mK}{\mathcal{K}}
\newcommand{\mI}{\nobreak\hspace{.16667em plus .08333em}\mathcal{I}}
\newcommand{\mM}{\mathcal{M}}
\newcommand{\mO}{\mathcal{O}}
\newcommand{\mP}{\mathcal{P}}
\newcommand{\mS}{\mathcal{S}}
\newcommand{\mT}{\mathcal{T}}
\newcommand{\mV}{\mathcal{V}}
\newcommand{\mW}{\mathcal{W}}

\newcommand{\U}{{\text{U}}}
\newcommand{\V}{{\text{V}}}


% Others
\newcommand{\cj}{\textcolor{blue}}
\newcommand{\pe}{\textcolor{magenta}}
\newcommand{\todo}{\textcolor{red}}
\newcommand{\cond}{\operatorname{cond}}
\newcommand{\sh}{\operatorname{sh}}
\newcommand{\ch}{\operatorname{ch}}
\newcommand{\interior}{\operatorname{int}}

\newtheorem{proposition}{Proposition}

\title{Boundary Reduction of Helmholtz Scattering Problem for exterior domains}
\author{Paul Escapil-Inchausp\'e}
%\date{}                                           % Activate to display a given date or no date


\begin{document}
\maketitle

Let $D$ be an open Lipschitz bounded domain, $D^c:=\IR^d\backslash \overline{D}$ and $\Gamma:=\partial D$, $d=2,3$ with exterior normal field $\bn$ pointing by convention toward the exterior domain. We use the superscript $i=1$ (resp.~$i=2$) when referring to $D^c$ (resp.~$D$). For a given scalar field $\U$, we introduce the Dirichlet (resp. Neumann) traces on $\Gamma$ as:
\be
\gamma_0 \U^i:= \U|_\Gamma \text{ and } \gamma_1 \U:= \langle \bn, \nabla \U|_\Gamma  \rangle.
\ee
For a given $\kappa$, we define the Helmholtz equation operator $L_\kappa:\U \mapsto \Delta \U + \kappa^2 \U$. For an index $\beta \in \{0,1\}$ us consider the following exterior boundary value problems {\bf (EP$_\beta$)}, which consist in finding the scattered field induced in response to a given incident field:

%%%%%%%%%%%%%%%%%%%%%%%%%%%%%%%%%%%%%%%%%%%%%% 
\subsubsection*{Exterior Problems}
\label{subsubs:EP_0}
%%%%%%%%%%%%%%%%%%%%%%%%%%%%%%%%%%%%%%%%%%%%%% 
Given $\kappa>0$ and $\U^{inc} \in H_{\text{loc}}^1(D^c)$ with $L_\kappa \U^{inc}=0$ in $D^c$, find $\U:=(\U^{scat}+\U^{inc}) \in H_{\text{loc}}^1(D^c)$ such that
\be
\left\{
\label{eq:EP_0}
\begin{array}{ll}
\Delta\U+\kappa^2\U  = 0&\text{ in } D^c,\\
\noalign{\vspace{3pt}}
  \gamma_{\beta} \U = 0&\text{ on } \Gamma,\\
  \noalign{\vspace{6pt}}
   \Big|\frac{\partial}{\partial r} \U^{scat}-i\kappa \U^{scat}\Big| = o(r^{\frac{1-d}{2}}) &\text{ for } r \to \infty.
\end{array}
\right.
\ee
Let us introduce the boundary integral operators $\mV_\kappa,\mK_\kappa$,$\mW_\kappa$,$\mK_\kappa',\mI$ defined on $\Gamma$.
Then, we have the following propositions:
\begin{proposition}
The \emph{sound-soft} problem ${\bf (EP_0)}$ is equivalent to each of the following BIE given on $\Gamma$:
\be
\left\{
\label{eq:B_EP_0}
\begin{array}{ll}
\mathcal{V}_\kappa \gamma_1 \U &= \gamma_0 \U^{inc}\text{ and }\\
\mathcal{V}_\kappa \gamma_1 \U &= -(\mathcal{K}_\kappa-\frac{1}{2}\mI) \gamma_0 \U^{inc}.
\end{array}
\right.
\ee
\end{proposition}

\begin{proposition}
The \emph{sound-hard} problem ${\bf (EP_1)}$ is equivalent to each of the following BIE given on $\Gamma$:
\be
\left\{
\label{eq:B_EP_0}
\begin{array}{ll}
\mathcal{W}_\kappa \gamma_0 \U &= \gamma_1 \U^{inc}\text{ and }\\
\mathcal{W}_\kappa \gamma_0 \U &= (\mK'_\kappa+\frac{1}{2}\mI) \gamma_1 \U^{inc}.
\end{array}
\right.
\ee
\end{proposition}

\end{document}  